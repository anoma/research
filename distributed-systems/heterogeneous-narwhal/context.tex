  % \tikzsetnextfilename{theIssue}
  % \begin{tikzpicture}
  %   \node[inner sep=0pt,outer sep=0pt] (header){%
  %   \faBoxOpen[duotone]
  % };
  %   \node[inner sep=0pt,outer sep=0pt,blue,anchor=west] (BlueCert) at (header.north east) {\faFileCertificate[duotone]};
  %   \node[inner sep=0pt,outer sep=0pt,red,anchor=east] (RedCert) at (header.north west) {\faFileCertificate[duotone]};
  %   \node[inner sep=0pt,outer sep=0pt,above left=4ex of header] (left) {\faBoxOpen[duotone]};
  %   \node[inner sep=0pt,outer sep=0pt,red,anchor=east] (RedCertLeft) at (left.north west) {\faFileCertificate[duotone]};
  %   \node[inner sep=0pt,outer sep=0pt,above right=4ex of header] (right) {\faBoxOpen[duotone]};
  %   \node[inner sep=0pt,outer sep=0pt,blue,anchor=west] (BlueCertRight) at (right.north east) {\faFileCertificate[duotone]};
  %   \draw (left.south) .. controls +(0,-.5) and +(0,+.8) .. (header);
  %   \draw (right.south) .. controls +(0,-.5) and +(0,+.8) .. (header);
  %   %   \node[above=1ex of header] (fork)  {\faIcon{fork}}; 
  % \end{tikzpicture}
\color{gray}%
\todo[inline,size=normalsize,caption={}]{.. a little bit of story telling, providing enough context and
  examples such that the reader can picture instances of the abstract concepts
  to be described %
  \begin{itemize}
  \item integrity vs.\ availability  (Some quotes from Charlotte in the endnotes.)
    \endnote{
      Some quotes from Charlotte:
      \begin{quote}
        different observers have their own
        beliefs about who might fail, and how
        \\…\\
        Blockchain applications implemented on Charlotte can
        refer to each other’s blocks
        \\…\\
        new mechanisms for \emph{entanglement} of different block histories
        \\…\\
        least
        ordering principle: only that which must be ordered, should be ordered
        \\…\\
        Authenticated Distributed Data Structure (ADDSs)
        \\…\\
        Charlotte ADDSs form the \emph{blockweb}
        \\…\\
        any \emph{observer} can specify their own threat model: %
        the failures they believe the system should securely tolerate. %
        \\…\\
        As a new abstraction layer, Charlotte specifies rules for formatting blocks, referencing
        blocks, and transmitting blocks across the network, and it also offers a formal model for reasoning
        about data integrity and availability.
        \\…\\
        Users can express their assumptions about server behavior (including possible failures)
        \\…\\
        Charlotte to serve as a […] general ADDS framework
        \\…\\
        Charlotte ADDSs can \emph{intersect}, or share blocks
        \\…\\
        “what is the least ordering we actually need?”
        \\…\\
        Charlotte provides a common framework for data structures from separate
        services to \ul{reference each other}.
        \\…\\
        Charlotte data structures are naturally composable: the \emph{union} of two data structures is itself a
        data structure
        \\…\\
        the \emph{intersection} of two data
        structures comprises the data that is part of both structures. We can think of \ul{cross-shard} transactions
        appended to a sharded blockchain ADDS as data in the intersection of multiple shard ADDSs.
      \end{quote}
    }%
  \item “learner-graph” jazz
  \end{itemize}
  The full details are deferred,
  \eg to the preliminaries. 
}


We want to overcome the seemingly opposing views on multi-chain vs.\ cross-chain worlds. %
We want to argue that there is a whole spectrum of ``worlds'' of which they are
but the extremes. %
Besides this general idea, %
we also pursue some specific \emph{qualitative} goals: %
\begin{itemize} %
\item every validator or user should be able to interact with any number of \base[s] of the ecosystems, %
  and %
\item there is a unique definitive state of every \base.
\end{itemize}
In technical terms, %
the person or service that requests to change or read the state of the \base[s]
is called a \emph{learner}. % 
One good example of learners are \emph{executor nodes} %
\cite{anomaSpecs}\todo{%
  check / update link, % 
  \href{https://specs.anoma.net/main/components/typhon/execution.html?highlight=exe\#execution}{execution} %
}, %
which are in charge of updating the state of one or several \base[s] %
according to a block of transactions that have been ordered by consensus. % 

For technical considerations,
the main points for validators are the following:
\begin{itemize}
\item %
  they can participate in the production of blocks for %
  as many \base[s] as they want %
\item %
  as long as they keep a record of relevant %
  transaction data or the ensuing state changes. %
\end{itemize}
The latter point roughly corresponds to the availability protocol %
while the former is mainly features in the integrity layer. %
\todo{discuss the general set up of the execution engine / executor nodes:
  these are the prime examples of learners we “have to” care about
}
Let us now look at turn to a concise recapitulation of the involved concepts. %

\color{black}